%---------- Inleiding ---------------------------------------------------------

\section{Introductie}%
\label{sec:introductie}

Deze bachelorproef zal vertrekken uit de probleemstelling die bachelor studenten toegepaste informatica in de afstudeerrichting systeem en netwerkbeheer aan Hogeschool Gent ondervinden bij het afstuderen. Namelijk de beperkte kennis omtrend cloud infrastructuur. Studenten die momenteel afstuderen beheersen onvoldoende de skills die nodig om direct te kunnen ingezet worden in bedrijven waar een hybride of volledige cloudomgeving aanwezig is. Deze bedrijven moeten vaak nog veel investeren in de student om kennis van cloud te verwerven.

Studenten die momenteel afstuderen verwerven tijdens hun studies te weinig cloud skills. AWS en Azure worden beperkt aangehaald doorheen het curriculum. Dit wil zeggen dat studenten  

Tijdens het doorlopen van het curriculum toegepaste informatica aan HOGENT valt op dat er vrij weinig cloud wordt aangeleerd. Dit heeft onder andere te maken met de kost die verbonden is aan het gebruiken van cloud. Er is slechts een beperkt aantal gratis resources beschikbaar bij de cloudplatformen zelf de dag van vandaag. Dit is meestal in vorm van gratis credits.

De resources die men aan Hogeschool Gent nu gebruikt zijn dus uitsluitend deze gratis educatieve credits. Die ontvangt men van de cloudplatformen zelf. Zoals het Azure for Students program waar studenten 100 dollar gratis kunnen gebruiken. Vergelijkbaar heeft Amazon AWS het AWS Education program. 

Het grote probleem hiermee is dat de credits vaak beperkt zijn in tijd en de hoeveelheid onvoldoende is om de principes van cloud aan te leren aan de studenten. Het inbrengen van extra credits als extra studiekost is een optie die overwogen kan worden, echter is deze niet geprefereerd. Dit om de betaalbaarheid van de studie te waarborgen.

De bedoeling van dit onderzoek zal zijn om in kaart te brengen welke alternatieven er bestaan om cloud aan te leren aan studenten. De theoretische resources die studenten kunnen wegwijs maken binnen cloud infrastructure vallen buiten de scope van dit onderzoek. Er zal in dit onderzoek enkel gekeken worden naar de praktische kant. Gezien het verwerven van deze skills in een bacheloropleiding ook hands on moet zijn. Er zal dus gekeken worden of self-hosting een oplossing kan zijn om cloud te simuleren en wat zijn hier de voor en nadelen van zijn. De bedoeling van dit onderzoek zou zijn een platform te kunnen opzetten met grote gelijkenissen aan de AWS en Azure infrastructure. 

Concluderen zal gebeuren met een aanbeveling naar de hogeschool Gent toe en een structureel overzicht van de alternatieven die verworven zijn gedurende het onderzoek. Dit alles moet een betere kijk geven op hoe cloud geimplementeerd kan worden in het de opleiding toegepaste informatica, afstudeerrichting systeem en netwerkbeheer. Zodat studenten die afstuderen beter voorbereid zijn op de huidige cloud en hybride infrastructuur die momenteel sterk aanwezig is in het werkveld. 

%Waarover zal je bachelorproef gaan? Introduceer het thema en zorg dat volgende zaken zeker duidelijk aanwezig zijn:

%\begin{itemize}
 % \item kaderen thema
 % \item de doelgroep
 % \item de probleemstelling en (centrale) onderzoeksvraag
 % \item de onderzoeksdoelstelling
%\end{itemize}

%Denk er aan: een typische bachelorproef is \textit{toegepast onderzoek}, wat betekent dat je start vanuit een concrete probleemsituatie in bedrijfscontext, een \textbf{casus}. Het is belangrijk om je onderwerp goed af te bakenen: je gaat voor die \textit{ene specifieke probleemsituatie} op zoek naar een goede oplossing, op basis van de huidige kennis in het vakgebied.

%De doelgroep moet ook concreet en duidelijk zijn, dus geen algemene of vaag gedefinieerde groepen zoals \emph{bedrijven}, \emph{developers}, \emph{Vlamingen}, enz. Je richt je in elk geval op it-professionals, een bachelorproef is geen populariserende tekst. Eén specifiek bedrijf (die te maken hebben met een concrete probleemsituatie) is dus beter dan \emph{bedrijven} in het algemeen.

%Formuleer duidelijk de onderzoeksvraag! De begeleiders lezen nog steeds te veel voorstellen waarin we geen onderzoeksvraag terugvinden.

%Schrijf ook iets over de doelstelling. Wat zie je als het concrete eindresultaat van je onderzoek, naast de uitgeschreven scriptie? Is het een proof-of-concept, een rapport met aanbevelingen, \ldots Met welk eindresultaat kan je je bachelorproef als een succes beschouwen?

%---------- Stand van zaken ---------------------------------------------------

\section{literatuurstudie}%
\label{sec:literatuurstudie}

Het onderzoeksdomein sluit aan bij de nog steeds actuele trend waarbij men volop inzet op cloud en hybride omgevingen. Het is dus belangrijk voor systeem en netwerkbeheerders voldoende kennis over de cloud te verwerven gedurende de studies. 

%Hier beschrijf je de \emph{state-of-the-art} rondom je gekozen onderzoeksdomein, d.w.z.\ een inleidende, doorlopende tekst over het onderzoeksdomein van je bachelorproef. Je steunt daarbij heel sterk op de professionele \emph{vakliteratuur}, en niet zozeer op populariserende teksten voor een breed publiek. Wat is de huidige stand van zaken in dit domein, en wat zijn nog eventuele open vragen (die misschien de aanleiding waren tot je onderzoeksvraag!)?

%Je mag de titel van deze sectie ook aanpassen (literatuurstudie, stand van zaken, enz.). Zijn er al gelijkaardige onderzoeken gevoerd? Wat concluderen ze? Wat is het verschil met jouw onderzoek?

%Verwijs bij elke introductie van een term of bewering over het domein naar de vakliteratuur, bijvoorbeeld~\autocite{Hykes2013}! Denk zeker goed na welke werken je refereert en waarom.

%Draag zorg voor correcte literatuurverwijzingen! Een bronvermelding hoort thuis \emph{binnen} de zin waar je je op die bron baseert, dus niet er buiten! Maak meteen een verwijzing als je gebruik maakt van een bron. Doe dit dus \emph{niet} aan het einde van een lange paragraaf. Baseer nooit teveel aansluitende tekst op eenzelfde bron.

%Als je informatie over bronnen verzamelt in JabRef, zorg er dan voor dat alle nodige info aanwezig is om de bron terug te vinden (zoals uitvoerig besproken in de lessen Research Methods).

% Voor literatuurverwijzingen zijn er twee belangrijke commando's:
% \autocite{KEY} => (Auteur, jaartal) Gebruik dit als de naam van de auteur
%   geen onderdeel is van de zin.
% \textcite{KEY} => Auteur (jaartal)  Gebruik dit als de auteursnaam wel een
%   functie heeft in de zin (bv. ``Uit onderzoek door Doll & Hill (1954) bleek
%   ...'')

%Je mag deze sectie nog verder onderverdelen in subsecties als dit de structuur van de tekst kan verduidelijken.

%---------- Methodologie ------------------------------------------------------
\section{Methodologie}%
\label{sec:methodologie}

Het onderzoek naar dit onderwerp zal gebeuren in 3 fases.

Waarbij eerst zal gekeken worden naar de alternatieven die er zijn voor het gewone educatieve creditsysteem. Er zal onderzocht worden welke, bij voorkeur self-hosted alternatieven er beschikbaar zijn en welke de beste gelijkheid kunnen bieden aan de actueel grootste spelers op de cloudmarkt. Dit zijnde AWS en Azure. 

Er zal een overzicht gemaakt worden van de belangrijkste alternatieven en deze zullen vergeleken worden. De focus in deze selectie zal liggen op de gelijkheid met AWS en Azure, alsook de betaalbaarheid en schaalbaarheid van dit product. 

Als laatste zal er een proof-of-concept worden opgezet met een self-hosted alternative.  Dit proof-of-concept zal beoordeeld worden op bruikbaarheid en gelijkheid met hedendaagse cloud instances. Er zal geavalueerd worden of deze toepassing schaalbaar zou zijn naar de volledige opleiding en of de meerwaarde voldoende groot zou zijn om dit toe te passen. 
%Hier beschrijf je hoe je van plan bent het onderzoek te voeren. Welke onderzoekstechniek ga je toepassen om elk van je onderzoeksvragen te beantwoorden? Gebruik je hiervoor literatuurstudie, interviews met belanghebbenden (bv.~voor requirements-analyse), experimenten, simulaties, vergelijkende studie, risico-analyse, PoC, \ldots?

%Valt je onderwerp onder één van de typische soorten bachelorproeven die besproken zijn in de lessen Research Methods (bv.\ vergelijkende studie of risico-analyse)? Zorg er dan ook voor dat we duidelijk de verschillende stappen terug vinden die we verwachten in dit soort onderzoek!

%Vermijd onderzoekstechnieken die geen objectieve, meetbare resultaten kunnen opleveren. Enquêtes, bijvoorbeeld, zijn voor een bachelorproef informatica meestal \textbf{niet geschikt}. De antwoorden zijn eerder meningen dan feiten en in de praktijk blijkt het ook bijzonder moeilijk om voldoende respondenten te vinden. Studenten die een enquête willen voeren, hebben meestal ook geen goede definitie van de populatie, waardoor ook niet kan aangetoond worden dat eventuele resultaten representatief zijn.

%Uit dit onderdeel moet duidelijk naar voor komen dat je bachelorproef ook technisch voldoen\-de diepgang zal bevatten. Het zou niet kloppen als een bachelorproef informatica ook door bv.\ een student marketing zou kunnen uitgevoerd worden.

%Je beschrijft ook al welke tools (hardware, software, diensten, \ldots) je denkt hiervoor te gebruiken of te ontwikkelen.

%Probeer ook een tijdschatting te maken. Hoe lang zal je met elke fase van je onderzoek bezig zijn en wat zijn de concrete \emph{deliverables} in elke fase?

%---------- Verwachte resultaten ----------------------------------------------
\section{Verwacht resultaat, conclusie}%
\label{sec:verwachte_resultaten}
Verwacht wordt dat er een self-hosted omgeving zal kunnen opgezet worden die de cloud systemen op een bepaald niveau zal kunnen weerspiegelen. In het beste geval zal dit een 1 op 1 weerspiegeling zijn. Echter zal dit uitgesloten zijn. Dit omdat het onmogelijk is een volledige kopie te verkrijgen van AWS of Azure. Beide providers laten namelijk geen self hosting van de servers toe. We trachten echter een zo goed mogelijke representatie van de cloud infrastructuur en principes te weerspiegelen. 

Bij het implementeren van de oplossing zou de meerwaarde van dit onderzoek bestaan voor studenten van de hogeschool en de hogeschool zelf. Daarnaast is dit ook een algemene toegevoegde waarde voor onze arbeidsmarkt en bedrijven die in een hybride of volledige cloudomgeving werken. 

Voor de studenten is de meerwaarde te vinden in de bredere kennis die men kan verwerven. Gezien cloud infrastrucutuur nog steeds een grote rol speelt bij bedrijven de dag van vandaag is het een grote meerwaarde om die kennis op zak te hebben.

Indien dit advies geimplementeerd wordt kan de hogeschool adverteren naar nieuwe studenten toe dat cloud uitgebreid aan bod komt in de opleiding. Iets wat nieuwe studenten kan aanspreken omwille van de actuele hype en belang in het werkveld rondom cloud infrastructure.

Bedrijven met een volledige of hybride cloudomgeving zullen een breder aanbod van studenten met cloudkennis kunnen verwerven op de startersmarkt. Er zal minder moeten geinvesteerd worden in de opleidingen om deze studenten de nodige basis te voorzien van cloud en de student zal sneller ingewerkt zijn in het bedrijf omwille van de voorkennis rond het gebruik van cloud infrastructuur.

%Hier beschrijf je welke resultaten je verwacht. Als je metingen en simulaties uitvoert, kan je hier al mock-ups maken van de grafieken samen met de verwachte conclusies. Benoem zeker al je assen en de onderdelen van de grafiek die je gaat gebruiken. Dit zorgt ervoor dat je concreet weet welk soort data je moet verzamelen en hoe je die moet meten.

%Wat heeft de doelgroep van je onderzoek aan het resultaat? Op welke manier zorgt jouw bachelorproef voor een meerwaarde?

%Hier beschrijf je wat je verwacht uit je onderzoek, met de motivatie waarom. Het is \textbf{niet} erg indien uit je onderzoek andere resultaten en conclusies vloeien dan dat je hier beschrijft: het is dan juist interessant om te onderzoeken waarom jouw hypothesen niet overeenkomen met de resultaten.

